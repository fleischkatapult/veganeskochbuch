\begin{recipe}{Haferflockenkekse, 35 Stück.}
	\item[Vorbereitung] Backofen auf 190\textcelsius Ober- und Unterhitze oder 175\textcelsius vorheizen.
	\ingredient{120\,mg\,Mehl}
	\ingredient{85\,g\,Haferflocken} und 
	\ingredient{1\,Msp\,Backpulver} in einer Schüssel gut miteinander vermischen. Die 
	\ingredient{150\,g\,Butter} schmelzen, leicht abkühlen lassen und mit 
	\ingredient{100\,g\,Zucker} und
	\ingredient{1\,Pck.\,Vanillezucker} und der
	\ingredient{1 Prise Salz} auf höchster Stufe gut 2 Minuten mit dem Handmixer schlagen. Wenn der Zucker sich aufgelöst hat nur noch mit dem Löffel arbeiten, den Handmixer nicht weiter verwenden. Nun die 
	\ingredient{5 EL Milch} und den 
	\ingredient{4 EL Sirup} unter die Butter-Zucker-Masse mischen und als letztes das Mehlgemisch unterheben. Alles so lange vermengen, bis das Mehl sich gut mit der Butter vermischt hat und ein einheitlicher Teig entstanden ist. Die Schüssel abdecken und für 15 Minuten kühl stellen. Nach dem Ruhen nochmals kurz durchrühren. Den Teig nun mit einem Teelöffel abstechen und kleine Teighäufchen auf ein mit Backpapier ausgelegtes Backblech setzen. 
	\hint Einen guten Abstand lassen, da die Kekse auseinander laufen. Ca. 5 cm reichten bei mir aus.
	Das Blech nun in den vorgeheizten Backofen geben und gute 10 Minuten backen. Nach Ende der Backzeit das Blech heraus nehmen und 2 Minuten stehen lassen. Dann das Backpapier vorsichtig, mit den Keksen darauf, vom Blech ziehen und die Kekse komplett auskühlen lassen. Am besten ist es, wenn man jeweils nur 1 Blech hinein gibt. Wer aber mit Umluft backt, der kann auch 2 Bleche hinein geben und sollte dann aufpassen, da die Kekse unterschiedlich schnell fertig sein könnten.
	\hint Sie sind schön crunchy und dem Original von Ikea sehr ähnlich, finde ich. 
	\preparationtime Etwa 60~Minuten.
	\washingup Normal.
\end{recipe}