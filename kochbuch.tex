\documentclass[12pt,a4paper]{article}
\usepackage[utf8x]{inputenc}
\usepackage[T1]{fontenc}
\usepackage[german]{babel}
\usepackage{cooking/cooking,textcomp}
\usepackage{csquotes}
\usepackage{url,hyperref}
\pagestyle{recipe}

\begin{document}

%%%%%%%%%%%%%%%%%%%%%%%%%%%%%%%%%%%%%%%%%
% Vertical Line Title Page 
% LaTeX Template
% Version 1.0 (27/12/12)
%
% This template has been downloaded from:
% http://www.LaTeXTemplates.com
%
% Original author:
% Peter Wilson (herries.press@earthlink.net)
%
% License:
% CC BY-NC-SA 3.0 (http://creativecommons.org/licenses/by-nc-sa/3.0/)
% 
% Instructions for using this template:
% This title page compiles as is. If you wish to include this title page in 
% another document, you will need to copy everything before 
% \begin{document} into the preamble of your document. The title page is
% then included using \titleGM within your document.
%
%%%%%%%%%%%%%%%%%%%%%%%%%%%%%%%%%%%%%%%%%

%----------------------------------------------------------------------------------------
%	PACKAGES AND OTHER DOCUMENT CONFIGURATIONS
%----------------------------------------------------------------------------------------

%\documentclass{book}

%\newcommand*{\plogo}{\fbox{$\mathcal{ANARCHIE}$}} % Generic publisher logo

%----------------------------------------------------------------------------------------
%	TITLE PAGE
%----------------------------------------------------------------------------------------

\newcommand*{\titleGM}{\begingroup % Create the command for including the title page in the document
\hbox{ % Horizontal box
\hspace*{0.2\textwidth} % Whitespace to the left of the title page
\rule{1pt}{\textheight} % Vertical line
\hspace*{0.05\textwidth} % Whitespace between the vertical line and title page text
\parbox[b]{0.75\textwidth}{ % Paragraph box which restricts text to less than the width of the page

{\noindent\Huge\bfseries Veganes Kochbuch}\\[2\baselineskip] % Title
{\large \textit{ Sammlung veganer Rezepte und mehr}}\\[4\baselineskip] % Tagline or further description
{\Large \textsc{Mohammed Lee}} % Author name

\vspace{0.5\textheight} % Whitespace between the title block and the publisher
%{\noindent The Publisher \plogo}\\[\baselineskip] % Publisher and logo
}}
\endgroup}

%----------------------------------------------------------------------------------------
%	BLANK DOCUMENT
%----------------------------------------------------------------------------------------

%\begin{document}

\pagestyle{empty} % Removes page numbers

\titleGM % This command includes the title page

%\end{document}

\section{Vorwort}
Dies ist ein Kochbuch mit veganen Rezepten, welches auf Github unter \enquote{veganeskochbuch} zu finden, kopieren und erweitern ist. Viel Spaß beim Mitmachen, Kochen und Genießen!

\clearpage
\section{Rezepte}
    \begin{recipe}{Spaghetti aus Zucchini}
	\ingredient{1 Zucchini} mit einem Schälmesser in feine Streifen schneiden, also in Spaghetti. In einem Topf mit
	\ingredient{Olivenöl} andünsten. 
	\ingredient{2 Tomaten} und
	\ingredient{Schwarze und grüne Oliven} dazu geben. Das die
	\ingredient{3 EL Mineralwasser} hinzufügen, damit das Gemüse etwas weicher wird und sich später um die Gabel wickeln lässt. Nun
	\ingredient{3 EL Tomatenmark} hinzugeben, würzen mit
	\ingredient{Salz} und
	\ingredient{Pfeffer}. Noch etwas köcheln lassen und fertig!
	\item{Schwierigkeitsgrad} simpel
	\preparationtime Etwa 25 Minuten.
	\washingup Normal.
\end{recipe}
    \begin{recipe}{Mangold-Spaghetti für 2 Personen}
	\item[Vorbereitung] Backofen auf 190\textcelsius Ober- und Unterhitze oder 175\textcelsius vorheizen.
	\ingredient{500 g Mangold} waschen und in Streifen schneiden. Statt nudellwasser eine ausreichende Menge an 
	\ingredient{Gemüsebrühe} zum Kochen bringen. 
	\ingredient{250 g Spaghetti} und Mangold zusammen kochen.
	\ingredient{1 Zehe Knoblauch} schälen, frein hacken und mit
	\ingredient{3 EL Olivenöl} vermischen. Nudeln und Gemüse abschütten und 
	\ingredient{2 EL Nudelwasser auffangen}. Die Nudeln mit dem Öl-Knoblauchgemich und den 2 EL Nudelwasser vermischen. Parmesan dazu geben.
	\item{Schwierigkeitsgrad} simpel
	\preparationtime Etwa 15 Minuten.
	\washingup Normal.
\end{recipe}
    \begin{recipe}{Haferflockenkekse, 35 Stück.}
	\item[Vorbereitung] Backofen auf 190\textcelsius Ober- und Unterhitze oder 175\textcelsius vorheizen.
	\ingredient{120\,mg\,Mehl}
	\ingredient{85\,g\,Haferflocken} und 
	\ingredient{1\,Msp\,Backpulver} in einer Schüssel gut miteinander vermischen. Die 
	\ingredient{150\,g\,Butter} schmelzen, leicht abkühlen lassen und mit 
	\ingredient{100\,g\,Zucker} und
	\ingredient{1\,Pck.\,Vanillezucker} und der
	\ingredient{1 Prise Salz} auf höchster Stufe gut 2 Minuten mit dem Handmixer schlagen. Wenn der Zucker sich aufgelöst hat nur noch mit dem Löffel arbeiten, den Handmixer nicht weiter verwenden. Nun die 
	\ingredient{5 EL Milch} und den 
	\ingredient{4 EL Sirup} unter die Butter-Zucker-Masse mischen und als letztes das Mehlgemisch unterheben. Alles so lange vermengen, bis das Mehl sich gut mit der Butter vermischt hat und ein einheitlicher Teig entstanden ist. Die Schüssel abdecken und für 15 Minuten kühl stellen. Nach dem Ruhen nochmals kurz durchrühren. Den Teig nun mit einem Teelöffel abstechen und kleine Teighäufchen auf ein mit Backpapier ausgelegtes Backblech setzen. 
	\hint Einen guten Abstand lassen, da die Kekse auseinander laufen. Ca. 5 cm reichten bei mir aus.
	Das Blech nun in den vorgeheizten Backofen geben und gute 10 Minuten backen. Nach Ende der Backzeit das Blech heraus nehmen und 2 Minuten stehen lassen. Dann das Backpapier vorsichtig, mit den Keksen darauf, vom Blech ziehen und die Kekse komplett auskühlen lassen. Am besten ist es, wenn man jeweils nur 1 Blech hinein gibt. Wer aber mit Umluft backt, der kann auch 2 Bleche hinein geben und sollte dann aufpassen, da die Kekse unterschiedlich schnell fertig sein könnten.
	\hint Sie sind schön crunchy und dem Original von Ikea sehr ähnlich, finde ich. 
	\preparationtime Etwa 60~Minuten.
	\washingup Normal.
\end{recipe}
	\begin{recipe}{Zwiebelmett}
	\ingredient{100\,g\,Naturreiswaffeln} in einer Schüssel klein zerbröseln, dann
	\ingredient{350\,ml\,lauwarmes Wasser} dazu geben und zum Brei verkneten. 
	\ingredient{2-3\,kleine\,Zwiebeln} in kleine Würfel schneiden und dazu geben, ebenso wie
	\ingredient{40g\,Tomatenmark} und 
	\ingredient{viel\,Salz} und 
	\ingredient{Pfeffer}. Alles zusammen mischen und mindestens 5 Stunden im Kühlschrank ziehen lassen
	\item[Quelle] http://www.veganguerilla.de/mett-mett-mett/
\end{recipe}
	\begin{recipe}{Onigiri (gefüllte Reisbällchen)}
	\ingredient{500\,g\,Basmati Reis}  in 
	\ingredient{2\,EL\,Gemüsebrühe} unter ständigem Rühren zum Kochen bringen, bis die Brühe verkocht und der Reisklebrig ist. Anschließend
	\ingredient{10\,EL\,Parmesan} unterrühren und erkalten lassen. Nun wird die Füllung vorbereitet:
	\ingredient{300\,g\,Kartoffeln} in Salzwasser weich kochen, abgießen, zerstampfen und erkalten lassen. 
	\ingredient{2\,Zwiebeln} in einer großen Pfanne mit
	\ingredient{Öl} weichdünsten. Dazu kommen zum mitdünsten 
	\ingredient{3\,grüne\,Chili} in feine Ringe geschnitten,
	\ingredient{1\,Knoblauchzehen}, fein gehackt und 
	\ingredient{5\,g Ingwer}, zerrieben oder sehr, sehr klein geschnitten. Danach kommen die Gewürze:
	\ingredient{3\,TL\,Garam\,Masala},
	\ingredient{1\,TL\,gemahlener\,Koriander},
	\ingredient{1\,TL\,gemahlener\,Kreuzkümmel},
	\ingredient{$\frac{1}{3}$\,TL\,gemahlener\,Kurkuma},
	\ingredient{$\frac{1}{3}$\,TL\,gemahlener\,Cayenne-Pfeffer} einrühren und darauf achten, dass genügend Öl in der Pfanne ist. Danach alles mit Wasser ablöschen, die Kartoffeln und 
	\ingredient{4\,EL\,frische oder tiefgekühlte Erbsen} sowie
	\ingredient{1\,$\frac{1}{3}$\,Salz} zugeben. Alles gut mischen, bei kleiner Hitze ziehen lassen und abkühlen lassen. Die Füllung sollte ein eher trockener Brei sein. Eine kleine Reiskugel formen, mit dem Daumen eindrücken, diese Delle mit der Füllung füllen und mit weiterem Reis verschließen. Die entstehende Kugel sollte so groß wie ein Tennisball sein.
\end{recipe}
	\begin{recipe}{Parmesan}
	\ingredient{1\,EL\,Cashewkerne} mit
	\ingredient{2\,EL\,Hefeflocken},
	\ingredient{$\frac{1}{2}$\,TL\,Salz},
	\ingredient{1\,Prise\,Pfeffer},
	\ingredient{1\,Prise\,Muskat},
	\ingredient{$\frac{1}{2}$\,geriebene\,Zitronenschale}	
\end{recipe}
	\begin{recipe}{Dinkelvollkorn-Pfannkuchen}
	\item[Personen] 4
	\ingredient{6\,Eiersatz} verrührt mit
	\ingredient{300\,g\,Vollkornmehl}
	\ingredient{400\,ml\,Sojamilch}
	\ingredient{1\,Tl\,Backpulver}
	\ingredient{$\frac{1}{2}$\,Tl\,Salz}
\end{recipe}

	\begin{recipe}{Vegan Burger, Würger?}
	\ingredient{1.\,Burgerhälfte} und
	\ingredient{Chilli-Ketchup}
	\ingredient{kleine Gurken}
	\ingredient{Tofu-Burger}
	\ingredient{Avokado-Mayo}
	\ingredient{Spinatblätter}
	\ingredient{Tomate}
	\ingredient{evtl. Kartoffelchips}
	\ingredient{2.\,Burgerhälfte}
\end{recipe}

	\begin{recipe}{Mávahlíðen, Bräter}
	\ingredient{1\,Paprika} und
	\ingredient{1\,Zwiebel} klein würfeln, gemeinsam anbraten.
	\ingredient{1\,handvoll\,Tofu} klein bröseln und mit
	\ingredient{2\,El\,Johannisbrotkernmehl}
	\ingredient{Paniermehl}
	\ingredient{Paprikagewürz}
	\ingredient{Chilligewürz} vermischen und abschmecken.
	\item[Zubereitung]{Alles zusammen mischen, kneten und zu Brätern formen und vor dem Grillen oder Braten mit Öl einpinseln.}
	\hint{Lieber etwas länger kneten, damit die bindende Wirkung des Johannisbrotkernmel auch wirkt.}
	\preparationtime{30~Minuten.}
	\washingup{mittel}
	\context{Rakete und nanooq haben dieses Rezept \enquote{komponiert} um Essen für die Reise nach Akureyri zu haben. War dann schon auf dem Hinweg alles aufgegessen.}
\end{recipe}

	\begin{recipe}{Reisbällchen}
	\ingredient{1 Knoblauchzehe} mit
	\ingredient{1 Zweibel} kleingehackt und in 
	\ingredient{Öl} andünsten.
	\ingredient{200\,g\,Reis} mit 
	\ingredient{1\,Gemüsebrühwürfel} matschig kochen.
	\ingredient{250\,g\,Kidneybohnen}
	\ingredient{1\,kleine\,Dose\,Mais} 
	\ingredient{2\,EL\,Haferflocken}
	\ingredient{Salz}
	\ingredient{Pfeffer}
	\ingredient{Paprikapulver}
	\ingredient{2\,El\,Tomatenmark} miteinander vermegen und mit Gabel matchig zerdrücken. 
	\item[Zubereitung] Alles miteinander vermischen und Reisbällchen in 
	\ingredient{Paniermehl} wälzen und in der Pfanne braten, bis sie goldbraun sind.
	\hint{Mindestens Tennisball große Bällchen machen.}
	\washingup{mittel}
\end{recipe}

	\begin{recipe}{Pythagoreer Suppe, 2 Personen}
	\ingredient{50\,g\,Polenta} in 
	\ingredient{100\,g Wasser} zubereiten und mit
	\ingredient{1 Eiersatz} vermengen, ziehen lassen und zu zwei Klößchen formen. 
	\modification{Die Klöße können auch angebraten werden}
	\ingredient{1 Gemüsebrühwürfel} in einem Kochtopf mit
	\ingredient{500\,ml\,Wasser} zum Kochen bringen.
	\item{Servieren} Suppe in zwei Schalen aufteilen und jeweils ein Kloß hinzufügen.
	\ingredient{Kontext} Diese Suppe erinnert an die Pythagoreer, die Schüler von Pythagoras von Samos. Diese haben sich \enquote{frugal} ernährt und waren wohl gute Mathematiker, leider ist wenig von ihnen erhalten geblieben. Aber sie waren nicht perfekt, ihren Freund Archytas haben sie im Meer ertränkt, weil er die Irrationalität von $ \sqrt{2} $ erkannt hat. Irrationale Zahlen war für sie Teufelswerk. Die Suppe nutzt diese Geschicht um zur Reflexion einzuladen. Wie gehen wir, die doch Gelegenheit haben so viel zu lernen, wissen und verändern, mit Dingen um, die uns neuartig sind? Außerdem erinnern wir uns an die Märtyrer der Wissenschaft. Archytas ist das Klößchen in der Suppe. Die Suppe ist für Feierlichkeiten geeignet, zum Beispiel dem Julfest. Die Idee entstammt dem Buch Anathem.
	\sidedish{Wasser oder Tee}
\end{recipe}

	\begin{recipe}{Suppe, Kartoffel und Linsen, für den Tag des Herren.}
	\ingredient{500\,ml\,Wasser} in einem Kochtopf mit
	\ingredient{1 Gemüsebrühwürfel} zum Kochen bringen. Währenddessen
	\ingredient{4\,handvoll\,Linsen} und
	\ingredient{2\,große\,Kartoffeln} ungeschält und sehr klein geschnitten (5x5~mm) dazugeben.
	\item[Zubereitung] Wenn die Kartoffeln weichgekocht sind, solange rühren, bis sie zerfallen und die Suppe dickflüssig wird.
	\hint Wasser bei Bedarf dazu geben. 
	\preparationtime Etwa 40~Minuten.
	\washingup Minimal.
\end{recipe}

	\begin{recipe}{Teig, für Pizza, Standard.}
	\item[Vorbereitung] Backofen auf 190\textcelsius~Ober- und Unterhitze vorheizen.
	\ingredient{600\,cc\,Vollkornmehl} in einer Schüssel mit
	\ingredient{360\,cc\,Weizenmehl}
	\ingredient{$\frac{1}{2}$\,Tl\,Zimtpulver}
	\ingredient{$\frac{1}{2}$\,Tl\,Backpulver}
	\ingredient{$\frac{1}{4}$\,Tl\,Salz} zu einem Teig verkneten. Dann separat
	\ingredient{240\,cc Ahornsirup} mit
	\ingredient{180\,cc Pflanzenbutter}
	\ingredient{$\frac{1}{2}$\,Tl\,Vanillepulver} mischen. Dann dazu
	\ingredient{$\frac{1}{2}$\,Tl\,Johannisbrotkernmehl} hinzufügen. Fertig ist die Geschmacksmischung.
	\item[Teig] Teig und Geschmacksmischung zusammen verkneten und mit 
	\ingredient{Zartbitter-Schokolade}
	\ingredient{Nüsse}
	\ingredient{Mandeln} abschmecken.
	\item[Backen] Auf einem Backblech den Keksteig verteilen und mittig in dem Ofen für 10~Minuten backen.
	\item[Zubereitungszeit] Etwa 25~Minuten.
	\item[Abwaschaufwand] Minimal.
	\item Ayumi Morohashi hat uns dieses Rezept gegeben. Es ist unser erstes \enquote{japanisches} Rezept.
\end{recipe}

	\begin{recipe}{Teig, für Pizza, Standard.}
	\item[Vorbereitung] Backofen auf 50\textcelsius~Ober- und Unterhitze vorheizen.
	\ingredient{500\,g\,Mehl} in einer Schüssel mit
	\ingredient{100\,ml\,Öl}
	\ingredient{1\,Tl\,Salz}
	\ingredient{1\,Tl\,Zucker}
	\ingredient{1\,El\,Hefe}
	\ingredient{$ n \approx 250$\,ml warmes Wasser} zum Teig kneten. 
	\hint Wasser bei Bedarf dazu geben. 
	\hint Mit einer Hand kneten.
	\item[Gehen lassen] Die Schüssel mit einem Tuch abdecken und in den Backofen stellen. Den Backofen ausschalten und den Teig mit der Restwärme 20~Minuten gehen lassen.
	\item[Pizza machen]
	\item[Backen] Pizze bei 200\textcelsius~Ober- und Unterhitze $\approx$ 35~Minuten backen.
	\item[Zubereitungszeit] Etwa 50~Minuten.
	\item[Abwaschaufwand] Minimal.
\end{recipe}
	\begin{recipe}{Brot, mit Hefe, Standard.}
	\item[Vorbereitung] Backofen auf 50\textcelsius~Ober- und Unterhitze vorheizen.
	\ingredient{500\,g\,Mehl} in einer Schüssel mit
	\ingredient{$\frac{1}{2}$\,El\,Hefe}
	\ingredient{1\,El\,Salz}
	\ingredient{1\,El\,Ahornsirup / Zucker}
	\ingredient{$ n \geq 200$\,ml warmes Wasser} zum Teig kneten. 
	\hint Wasser bei Bedarf dazu geben. 
	\hint Mit einer Hand kneten.
	\item[Gehen lassen] Die Schüssel mit einem Tuch abdecken und in den Backofen stellen. Den Backofen ausschalten und den Teig mit der Restwärme 30\,Minuten gehen lassen.
	\item[Backen] Die Schüssel aus dem Ofen nehmen und Ober- und Unterhitze auf 200\textcelsius~schalten. Den Teig nochmal durchkneten und in eine Form oder auf ein Blech mit Backpapier mittig in den Ofen schieben. 
	\hint Hier kann man noch weitere Zutaten in den Teig einkneten: Samen, zerkleinerte Nüsse, klein geschnittenes Gemüse.
	\hint Eine Tasse mit Wasser neben den Teig stellen, soll eine besonders krustige Kruste machen.
	\item[Runter schalten] Nach 15\,Minuten die Ofentemperatur auf 180\textcelsius~runter schalten und ungefähr 45 Minuten backen.
	\hint In einem Jute-Beutel hält sich das Brot am längsten lecker.
	\item[Zubereitungszeit] Etwa 60~Minuten.
	\item[Abwaschaufwand] Minimal.
\end{recipe}

	\begin{recipe}{Spätzle, nackt ohne alles.}
\ingredient{500\,g\,Mehl} in einer Schüssel und Schneebesen mit
\ingredient{20\,g\,Salz}
\ingredient{6\,Eier\,Eiersatz}
\ingredient{200\,Wasser} zum Teig vermengen.
\item[Teig] Teig 15\,Minuten ziehen lassen und in der Zwischenzeit Topf mit Wasser zum Kochen bringen. 
\item[Schaben \& Schöpfen] Den Teig auf ein Brett kippen und über den Topf halten. Mit einem Messer, welches eine ungebogene Klinge hat, ein Streifen in das kochende Wasser abschaben. Oben schwimmende Spätzle können abgeschöpft werden. 
\hint Wenn abgeschöpfte Spätzle mit kaltem Wasser abgeschreckt werden, kleben sie nicht aneinander.
\hint Wahlweise können die Spätzle nun auch angebraten werden.
\item[Verarbeitung] Da muss auf jeden Fall noch eine Sauce zu. Weil die Spätzle geschmacksneutral sind, kann man auch eine süße Sauce verwenden.
\preparationtime Etwa 25~Minuten.
\washingup Minimal.
\item Die dazu notwendige Soße hat meine Freunden gezaubert. Ein Saucenteil soll auch noch kommen.
\end{recipe}

	\begin{recipe}{Schlechter Meta-Witz}
	\ingredient{1\,Github-Projekt} über einige Tage leer lassen.
	\item[Reaktion] Silsha fragt, ob ein leeres Projekt über vegane Rezepte einen schlechten Meta-Witz darstellt. Wahlweise silsha schmeicheln und auf andere Gedanken bringen.
\end{recipe}

\clearpage	
\subsection[Saucen]{Saucen}
	\begin{recipe}{Habanero Mango Sauce}
\ingredient{120\,ml\,Apfelweinessig},
\ingredient{120\,ml\,Limettenessig oder Zitronensaft} und
\ingredient{1\,Schuss\,Orangensaft} aufkochen und Gewürze hinzugeben:
\ingredient{1\,TL\,Thymian} und
\ingredient{$\frac{1}{2}$\,Kurkuma}.
\ingredient{250g\,Habanero-Chilli} ohne Stengel, 
\ingredient{1\,Zwiebel},
\ingredient{2\,Knoblauchzehen}, 
\ingredient{2\,Mangos} und
\ingredient{1\,Tomate} kleinschneiden und püriren. Dazu die vorhin aufgekochte Mischung aus Saft und Gewürzen geben, ordentlich mischen und in kleine Flaschen abfüllen.
\hint{} Zum Haltbarmachen die Flaschen offen in einem ofenfesten Topf in etws 3 cm tiefes Wasser stellen und bei 160\textcelsius~ im Ofen erhitzen. Wenn in der Flasche Bläschen aufsteigen, Ofen abschalten und nach 15 bis 20 Minuten Flaschen verschließen.
\end{recipe}
    \begin{recipe}{Avocado-Mayo.}
\ingredient{1\,Avocado} ohne Schale und Kern in eine Schüssel mit
\ingredient{1\,Prise\,Salz}
\ingredient{1\,Prise\,Pfeffer}
\ingredient{Cashewmus oder weißes Mandelmus} vermischen. Dann mit
\ingredient{etwas Zitronensaft} abschmecken.
\end{recipe}
    \begin{recipe}{Chilli-Ketchup.}
	\ingredient{Tomatenmark} in einer Schüssel mit
	\ingredient{1\,Prise\,Salz}
	\ingredient{1\,Prise\,Pfeffer}
	\ingredient{Chilli-Flocken}
	\ingredient{Abrieb einer Zitronenschale}
	\ingredient{Ahornsirup} abschmecken.
\end{recipe}
	
\clearpage
\section{Vegane Restaurants}
    \subsection{Mannheim, Germany}
\begin{enumerate}
 \item Heller's Vegetarisches Restaurant \& Café, N 7, 13-15, 68161 Mannheim. \href{http://hellers-restaurant.de/}{http://hellers-restaurant.de/}
\end{enumerate}
    \subsection{Siegen, Germany}
\begin{enumerate}
	\item Essbar, Kohlbettstrasse 20, 57072 Siegen, Tel: 0271 220 123 80, Fax 0170 5357497
	\item Pizza Ciao, Alte Poststr. 19, 57072 Siegen, Tel: 0271 57496, www.pizza-ciao-siegen.de. Mo. - Sa. 11:30 - 21:30. Sonn und Feiertags geschlossen. Von Mai bis September an Sonnund Feiertagen ab 18 Uhr geöffnet
\end{enumerate}
	\subsection{Reykjavik, Island}
\begin{enumerate}
	\item Extasy's Heart-Garden, Klapparstigur 37
	\item C is for Cookie, Tysgate 8
	\item Cafe Babalu, Solavoroustigur 22a
	\item Glo - Engiateigur 19
	\item Glo Lauavegi 20b
	\item Kryddlegin Hjortu, Hverfisgate 33
	\item Nudluskalin, Skolavoroustig 8
\end{enumerate}

	
\clearpage
\section{HashMap}
	\begin{enumerate}
	\item[Eier] Bei Eiern kann man Leinsamen in Wasser aufweichen oder Johanniskernmehl oder "No-Egg"-Pulver nehmen. 
	\item[Milch]  Milch von der Kuh, kann man auch aus Reis, Hafer, Mandeln, Nüssen und Soja machen. 
\end{enumerate}
	
\end{document}
\endinput
%%
%% End of file `kochbuch.tex'.
